\documentclass[12pt]{article}
\usepackage[utf8]{inputenc}
\usepackage[margin=1in]{geometry}
\usepackage[document]{ragged2e}
\setlength{\parskip}{1em}

\usepackage[T1]{fontenc}
\usepackage{libertine}

\begin{document}

As a master's student currently studying at the University of Pennsylvania, I am
deeply drawn by the academic environment at Penn and I would like to apply for
the doctoral program in computer and information science, focusing on
distributed systems, networks, and operating systems.

My interest in these fields dates back to my undergraduate time when I was
struck by how operating and network systems designed for the general-purpose
machine have extended the computing power to many other subjects, such as CAD
simulation in civil engineering. The computer system which powers up data
centers and cloud platforms has become a vital infrastructure that we almost
cannot live without and is still playing an important role in reshaping, and
eventually, changing the world. Despite varieties of applications, there are
still many research hot spots such as novel structures for software security,
improving the scale and performance of data warehouses, and adaptation for
upcoming technologies like deep learning and virtual reality. It is my very
intention to devote myself to such an inspiring area and build up an academic
career.

Motivated by such interest, I joined a dual degree program in computer and
information science at Penn. Many intensive program course materials combined
with practical orientated projects have provided me with solid background
knowledge in most major topics in computer science, as well as programming and
problem-solving skills. We were asked most of the time to implement from scratch
what we have learned in the class. For example, we implemented a Penn-OS in C
with a shell, scheduler, memory management unit, and a FAT file system with
minimum library functions allowed in CIS548 to gain a full picture of different
components of the operating system. We also designed a relational database that
could be deployed to the web with a self-programmed front-end through AWS in
CIS550, and a simple network protocol similar to RTSP to stream music in
real-time with the corresponding client and server-side implemented by ourselves
in CIS553. I learned deeply about OS design and implementation, database and
information systems, as well as networked systems, and many other subjects in
computer science. And it could be reflected from my grades that I put a lot of
effort to excel these courses and to get the best of them.

I have also participated in scientific research to gain experience in academic
skills. I joined professor Boon Thau Loo's lab in the distributed
information-centric system as a research assistant from May 2020 till August
2021. During my RA time, I worked closely with Dr. Nik Sultana, a previous
postdoc at Penn, and Henry Zhu, a Penn graduate currently working at Amazon.
Focusing mainly on software compartmentalization, I worked on several projects
regarding debugging, serializing, and re-configuring existing software in a
distributed fashion. I created a front-end for the debugger designed for
compartmentalized software and the front-end can easily control and communicate
with multiple processes corresponding to the compartments that require
debugging. I also contributed to the implementation of a serializer designed for
C-style data structures, which is used in generating required data from memory
to pass between different compartments. By analyzing with Valgrind, I was able
to locate memory leaks and corrected the serializer's garbage collection
behavior corresponding to different data structures like a linked list or binary
tree.

The project to which I put the most effort during my RA time is re-configuring
software in a distributed fashion with a novel architecture named C-SAW. We
analyzed several popular software, such as Redis and cURL, and focused on how to
improve them by introducing state-management abstraction. My work in this
project resides mainly on re-configuring cURL and the overhead evaluation of the
modified version of cURL and Redis. To make such modifications on cURL, I
analyzed the general design of cURL, identified the data structure to monitor,
and serialized it with the previously mentioned tool. By modifying the source
and integrating it with a compartmentalization library, I was able to implement
constant remote auditing of cURL's download speed in separated machines. In
terms of evaluation, I designed a whole set of scripts to automate the testing
process. The new cURL was compared against the original version in terms of
execution time. Newly implemented features in Redis, such as fail-over and
sharding, were also tested through query speed variation and cumulative
distribution functions of each shard. Our results regarding this novel structure
were very exciting, and the resulting paper was submitted for review.

I gained a lot in many ways from my RA experience, and it will no doubt be a
unique strength of mine in a doctoral program. I can think and act more
analytically and independently when facing academic problems, which often do not
have existing solutions to lookup. Many unexpected behaviors in pipelining when
retrofitting large systems like cURL taught me to use tools such as GDB in
multiprocess debugging and GNU gprof to analyze unprecedented situations and
identify possible factors causing the trouble. And I was also introduced to
various comparison measurements and required to figure out a compelling way to
evaluate different types of features when designing the evaluation scripts for
Redis. The 2020 pandemic has also leveled the difficulty of communication, and
this RA experience greatly improved my ability to work in a team and communicate
actively with other members.

After the beginning of my master's program, I enjoyed every day spend at Penn
and it is my sincere hope to continue my academic development here. I believe
the Ph.D. program of CIS at Penn is particularly well-aligned with my interests
and I would like to continue working with Prof. Boon Thau Loo and Vincent Liu.
In the Flightplan project, Prof. Boon adopted a similar logic as I have
experienced to dis-aggregate P4 programs and map them to several, possibly
heterogeneous, dataplanes, which could empower its computing capability in a
single target's recourse. Their recent collaborative work on evaluating the
performance overhead of dis-aggregated data centers has also shown many
potential research opportunities, such as DDC-aware buffer pool policy, new cost
models for query planning and optimization, and improving parallel DBMS to avoid
unnecessary data shuffling. Given my experiences in academic research, being a
Ph.D. student at Penn would be very ideal for my academic career. I strongly
believe that this program will fulfill my passion for distributed and network
systems, and allow me to leverage my background and skills to contribute to the
program's diversity.

\end{document}
